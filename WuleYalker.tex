\documentclass[12pt,letterpaper]{article}
\usepackage{amsmath, amssymb, graphics}
\usepackage{fullpage}

\newcommand{\mathsym}[1]{{}}
\newcommand{\unicode}{{}}

\begin{document}

\section*{Derivation and solution of the ``Wule--Yalker'' equations}

These equations are nothing but the Yule-Walker equations rearranged to that
autocorrelations might be recovered from autoregressive process parameters.

\subsection*{Setting}

The standard AR(p) process is taken as
\[
x_n+a_1x_{n-1}+\dots+a_px_{n-p}=\epsilon _n
\]
in which the \(a_i\) are called the ``parameters" and \(\epsilon _n\sim
N\left(0,\sigma _{\epsilon }{}^2\right)\). By multiplying by \(x_{n-k}\) and
taking the expectation, one arrives at the classical Yule Walker equations:
\begin{align*}
1+a_1\rho ^1+\dots+a_p\rho ^p &= \frac{\sigma _{\epsilon }{}^2}{\sigma _x{}^2},
&
k&=0
\\
\rho ^k+a_1\rho ^{k-1}+\dots+a_p\rho ^{k-1}&=0
&
k &> 0
\end{align*}
Here $\rho^i$ denotes the lag $i$ autocorrelation which are symmetric about lag
$0$. These equations are solved by a variant of the Levinson-Durbin to estimate
parameters from observed autocorrelations. We want the converse--- computing
autocorrelations $\rho^i$ given parameters $a^i$.  For $k\geq p$ this is simple
as the latter, iterative relation can be used. However, For $0<k<p$ the
equations form a linear system behaving as a ``boundary condition''.  Many
authors solve this system explicitly for only $p=1$ and $p=2$ and consider the
matter closed.  We seek a systematic way to derive and solve the boundary
condition system, which we call the ``Wule-Yalker'' equations to hint at their
straightforward heritage.

\subsection*{Details}

Several concrete examples of the linear system, derived mechanically using
Mathematica, are now provided. It is nothing but iterating and rearranging the
Yule--Walker $k>0$ relationship considering $\rho ^i$ to be unknown and using
that $\rho^0=1$.

\begin{quote}
\noindent\({\text{Unprotect}[\text{Power}];}\\
{\rho ^{\text{k$\_$}}\text{:=}\rho ^{-k}\text{/;}k<0;}\\
{a_0=1;}\)
\end{quote}

Next come expressions for getting the \(k^{\text{th}}\) equation for given \(p\), collecting all relations for \(1\leq k\leq p\) into a list, extracting
the coefficients on different \(\rho ^i\) terms, obtaining the matrix of coefficients, obtaining the vector of data, and preparing them both for
display.

\begin{quote}
\noindent\({\text{yw}[\text{p$\_$},\text{k$\_$}]\text{:=}\text{Sum}\left[a_i\rho ^{k-i},\{i,0,p\}\right];}\\
{\text{yt}[\text{p$\_$}]\text{:=}\text{Collect}[\text{Table}[\text{yw}[p,k],\{k,1,p\}],\rho ];}\\
{\text{yc}[\text{p$\_$}]\text{:=}\text{PadRight}[\text{CoefficientList}[\text{yt}[p],\rho ]];}\\
{\text{ym}[\text{p$\_$}]\text{:=}\text{Drop}[\text{yc}[p],\text{None},1];}\\
{\text{yd}[\text{p$\_$}]\text{:=}-\text{FullSimplify}[\text{Take}[\text{yc}[p],\text{All},1]];}\\
{\text{yp}[\text{p$\_$}]\text{:=}\text{Thread}[\text{MatrixForm}[\{\text{ym}[p],\text{yd}[p]\}]];}\)
\end{quote}

Examine the matrix of coefficients and vector of data for small \(p\):

\begin{quote}
\noindent\({\text{Do}[}\\
{\text{Print}[\text{$\texttt{"}$p=$\texttt{"}$},p];}\\
{\text{Print}[\text{yp}[p]],\{p,1,8\}}\\
{]}\\
{ }\)
\end{quote}

\[
\begin{bmatrix}
 1
\end{bmatrix}
\begin{bmatrix}
 \rho^1
\end{bmatrix}
=
\begin{bmatrix}
 -a_1
\end{bmatrix}
\]

\[
\begin{bmatrix}
 1+a_2 & 0 \\
 a_1 & 1
\end{bmatrix}
\begin{bmatrix}
 \rho^1 \\
 \rho^2
\end{bmatrix}
=
\begin{bmatrix}
 -a_1 \\
 -a_2
\end{bmatrix}
\]

\[
\begin{bmatrix}
 1+a_2 & a_3 & 0 \\
 a_1+a_3 & 1 & 0 \\
 a_2 & a_1 & 1
\end{bmatrix}
\begin{bmatrix}
 \rho^1 \\
 \rho^2 \\
 \rho^3
\end{bmatrix}
=
\begin{bmatrix}
 -a_1 \\
 -a_2 \\
 -a_3
\end{bmatrix}
\]

\[
\begin{bmatrix}
 1+a_2 & a_3 & a_4 & 0 \\
 a_1+a_3 & 1+a_4 & 0 & 0 \\
 a_2+a_4 & a_1 & 1 & 0 \\
 a_3 & a_2 & a_1 & 1
\end{bmatrix}
=
\begin{bmatrix}
 \rho^1 \\
 \rho^2 \\
 \rho^3 \\
 \rho^4
\end{bmatrix}
\begin{bmatrix}
 -a_1 \\
 -a_2 \\
 -a_3 \\
 -a_4
\end{bmatrix}
\]

\[
\begin{bmatrix}
 1+a_2 & a_3 & a_4 & a_5 & 0 \\
 a_1+a_3 & 1+a_4 & a_5 & 0 & 0 \\
 a_2+a_4 & a_1+a_5 & 1 & 0 & 0 \\
 a_3+a_5 & a_2 & a_1 & 1 & 0 \\
 a_4 & a_3 & a_2 & a_1 & 1
\end{bmatrix}
\begin{bmatrix}
 \rho^1 \\
 \rho^2 \\
 \rho^3 \\
 \rho^4 \\
 \rho^5
\end{bmatrix}
=
\begin{bmatrix}
 -a_1 \\
 -a_2 \\
 -a_3 \\
 -a_4 \\
 -a_5
\end{bmatrix}
\]

\[
\begin{bmatrix}
 1+a_2 & a_3 & a_4 & a_5 & a_6 & 0 \\
 a_1+a_3 & 1+a_4 & a_5 & a_6 & 0 & 0 \\
 a_2+a_4 & a_1+a_5 & 1+a_6 & 0 & 0 & 0 \\
 a_3+a_5 & a_2+a_6 & a_1 & 1 & 0 & 0 \\
 a_4+a_6 & a_3 & a_2 & a_1 & 1 & 0 \\
 a_5 & a_4 & a_3 & a_2 & a_1 & 1
\end{bmatrix}
\begin{bmatrix}
 \rho^1 \\
 \rho^2 \\
 \rho^3 \\
 \rho^4 \\
 \rho^5 \\
 \rho^6
\end{bmatrix}
=
\begin{bmatrix}
 -a_1 \\
 -a_2 \\
 -a_3 \\
 -a_4 \\
 -a_5 \\
 -a_6
\end{bmatrix}
\]

\[
\begin{bmatrix}
 1+a_2 & a_3 & a_4 & a_5 & a_6 & a_7 & 0 \\
 a_1+a_3 & 1+a_4 & a_5 & a_6 & a_7 & 0 & 0 \\
 a_2+a_4 & a_1+a_5 & 1+a_6 & a_7 & 0 & 0 & 0 \\
 a_3+a_5 & a_2+a_6 & a_1+a_7 & 1 & 0 & 0 & 0 \\
 a_4+a_6 & a_3+a_7 & a_2 & a_1 & 1 & 0 & 0 \\
 a_5+a_7 & a_4 & a_3 & a_2 & a_1 & 1 & 0 \\
 a_6 & a_5 & a_4 & a_3 & a_2 & a_1 & 1
\end{bmatrix}
\begin{bmatrix}
 \rho^1 \\
 \rho^2 \\
 \rho^3 \\
 \rho^4 \\
 \rho^5 \\
 \rho^6 \\
 \rho^7
\end{bmatrix}
=
\begin{bmatrix}
 -a_1 \\
 -a_2 \\
 -a_3 \\
 -a_4 \\
 -a_5 \\
 -a_6 \\
 -a_7
\end{bmatrix}
\]

\[
\begin{bmatrix}
 1+a_2 & a_3 & a_4 & a_5 & a_6 & a_7 & a_8 & 0 \\
 a_1+a_3 & 1+a_4 & a_5 & a_6 & a_7 & a_8 & 0 & 0 \\
 a_2+a_4 & a_1+a_5 & 1+a_6 & a_7 & a_8 & 0 & 0 & 0 \\
 a_3+a_5 & a_2+a_6 & a_1+a_7 & 1+a_8 & 0 & 0 & 0 & 0 \\
 a_4+a_6 & a_3+a_7 & a_2+a_8 & a_1 & 1 & 0 & 0 & 0 \\
 a_5+a_7 & a_4+a_8 & a_3 & a_2 & a_1 & 1 & 0 & 0 \\
 a_6+a_8 & a_5 & a_4 & a_3 & a_2 & a_1 & 1 & 0 \\
 a_7 & a_6 & a_5 & a_4 & a_3 & a_2 & a_1 & 1
\end{bmatrix}
\begin{bmatrix}
 \rho^1 \\
 \rho^2 \\
 \rho^3 \\
 \rho^4 \\
 \rho^5 \\
 \rho^6 \\
 \rho^7 \\
 \rho^8
\end{bmatrix}
=
\begin{bmatrix}
 -a_1 \\
 -a_2 \\
 -a_3 \\
 -a_4 \\
 -a_5 \\
 -a_6 \\
 -a_7 \\
 -a_8
\end{bmatrix}
\]

Evidently, the matrices are the sum of two contributions. The first is a
full-rank, unit-lower-triangular Toeplitz matrix. The second is a
rank-deficient Hankel matrix. In the above $p=8$ case, e.g., the decomposition
is
\[
\left(
\begin{bmatrix}
 1 & 0 & 0 & 0 & 0 & 0 & 0 & 0 \\
 a_1 & 1 & 0 & 0 & 0 & 0 & 0 & 0 \\
 a_2 & a_1 & 1 & 0 & 0 & 0 & 0 & 0 \\
 a_3 & a_2 & a_1 & 1 & 0 & 0 & 0 & 0 \\
 a_4 & a_3 & a_2 & a_1 & 1 & 0 & 0 & 0 \\
 a_5 & a_4 & a_3 & a_2 & a_1 & 1 & 0 & 0 \\
 a_6 & a_5 & a_4 & a_3 & a_2 & a_1 & 1 & 0 \\
 a_7 & a_6 & a_5 & a_4 & a_3 & a_2 & a_1 & 1
\end{bmatrix}
+
\begin{bmatrix}
 a_2 & a_3 & a_4 & a_5 & a_6 & a_7 & a_8 & 0 \\
 a_3 & a_4 & a_5 & a_6 & a_7 & a_8 & 0 & 0 \\
 a_4 & a_5 & a_6 & a_7 & a_8 & 0 & 0 & 0 \\
 a_5 & a_6 & a_7 & a_8 & 0 & 0 & 0 & 0 \\
 a_6 & a_7 & a_8 & 0 & 0 & 0 & 0 & 0 \\
 a_7 & a_8 & 0 & 0 & 0 & 0 & 0 & 0 \\
 a_8 & 0 & 0 & 0 & 0 & 0 & 0 & 0 \\
 0 & 0 & 0 & 0 & 0 & 0 & 0 & 0
\end{bmatrix}
\right)
\begin{bmatrix}
 \rho^1 \\
 \rho^2 \\
 \rho^3 \\
 \rho^4 \\
 \rho^5 \\
 \rho^6 \\
 \rho^7 \\
 \rho^8
\end{bmatrix}
=
\begin{bmatrix}
 -a_1 \\
 -a_2 \\
 -a_3 \\
 -a_4 \\
 -a_5 \\
 -a_6 \\
 -a_7 \\
 -a_8
\end{bmatrix}
.
\]

This special structure is not surprising given that the Yule-Walker system is
symmetric Toeplitz. Gohberg and Koltracht 1989 presented an ``Efficient
Algorithm for Toeplitz Plus Hankel Matrices'' scaling like $5N^2$ operations.
Unfortunately the article details only the case for a real, symmetric sum
stemming from a real, symmetric Toeplitz summand. They hinted that the
non-symmetric algorithm, which we desire, is similar.

\subsection*{TODO}

\begin{enumerate}
 \item Actually write the general, closed-form expression for these matrices.
 \item Find or derive the algorithm for solving with a real-valued operator
  that is the sum of a non-symmetric Toeplitz matrix and a Toeplitz matrix.
\end{enumerate}

\subsection*{Comments}

Of course, as suggested by Broersen 2006 in ``Automatic Autocorrelation and
Spectral Analysis'' section 5.4, we could obtain the process reflection
coefficients from the parameters and then the autocorrelations from the
reflection coefficients, but what fun would that be?  Presumably any efficient
solution we find will be equivalent to Broersen's suggested two-pass solution
solution.

\end{document}
